\documentclass[11pt]{article}
\usepackage{tikz}
\usepackage{pifont}
\usepackage{amsmath}
\usepackage{marvosym}
\usepackage{verbatim}
\usepackage{listings}
\usepackage[utf8]{inputenc}
\usepackage{subfig}
\usepackage[switch,columnwise]{lineno}
\usepackage{amssymb}
\usepackage{enumitem}
\usepackage[nottoc,numbib]{tocbibind}
   
\usepackage{hyperref}
\hypersetup{
bookmarks=false,         % show bookmarks bar?
unicode=true,          % non-Latin characters in Acrobat's bookmarks
pdftoolbar=true,        % show Acrobat's toolbar?
pdfmenubar=true,        % show Acrobat's menu?
pdffitwindow=true,     % window fit to page when opened
pdfstartview={FitH},    % fits the width of the page to the window
pdftitle={Node allocators},    % title
pdfauthor={Marcelo Zimbres},     % author
pdfsubject={C++ allocators},   % subject of the document
pdfcreator={Marcelo Zimbres},   % creator of the document
pdfproducer={Marcelo Zimbres}, % producer of the document
pdfkeywords={allocators} {C++}, % list of keywords
pdfnewwindow=true,      % links in new window
colorlinks=true,        % false: boxed links; true: colored links
linkcolor=red,          % color of internal links
citecolor=red,        % color of links to bibliography
filecolor=red,      % color of file links
linktocpage=true,
urlcolor=blue           % color of external links
}

\lstset{
  language=C++,                 % the language of the code
  backgroundcolor=\color{white},   % choose the background color; you must add \usepackage{color} or \usepackage{xcolor}
  basicstyle=\footnotesize,        % the size of the fonts that are used for the code
  breakatwhitespace=false,         % sets if automatic breaks should only happen at whitespace
  breaklines=true,                 % sets automatic line breaking
  keywordstyle=\color{blue},       % keyword style
  captionpos=b,                    % sets the caption-position to bottom
  commentstyle=\color{blue},       % comment style
  deletekeywords={},            % if you want to delete keywords from the given language
  escapeinside={\%*}{*)},          % if you want to add LaTeX within your code
  extendedchars=true,              % lets you use non-ASCII characters; for 8-bits encodings only, does not work with UTF-8
  frame=single,                    % adds a frame around the code
  keepspaces=true,                 % keeps spaces in text, useful for keeping indentation of code (possibly needs columns=flexible)
  morekeywords={using, static_assert},            % if you want to add more keywords to the set
  numbers=left,                    % where to put the line-numbers; possible values are (none, left, right)
  numbersep=10pt,                   % how far the line-numbers are from the code
  numberstyle=\tiny\color{black}, % the style that is used for the line-numbers
  rulecolor=\color{black},         % if not set, the frame-color may be changed on line-breaks within not-black text (e.g. comments (green here))
  showspaces=false,                % show spaces everywhere adding particular underscores; it overrides 'showstringspaces'
  showstringspaces=false,          % underline spaces within strings only
  showtabs=false,                  % show tabs within strings adding particular underscores
  stepnumber=1,                    % the step between two line-numbers. If it's 1, each line will be numbered
  stringstyle=\color{red},     % string literal style
  tabsize=2,                       % sets default tabsize to 2 spaces
  title=\lstname                   % show the filename of files included with \lstinputlisting; also try caption instead of title
}

\colorlet{blah}{brown!60!black} % box color

\begin{document}

\date{}
\title{\bf Splitting {\it node} and {\it array} allocation in allocators}
\maketitle
\noindent
%\vspace{-1cm} \\
{\bf Document number}:  P0310R0 \\
{\bf Date}:  2016-03-19\\ % \today 
{\bf Project}: Programming Language C++ \\
{\bf Audience}: Library Evolution Working Group \\
{\bf Reply to}: Marcelo Zimbres (\href{mailto:mzimbres@gmail.com}{mzimbres@gmail.com}) 

\vspace{1cm}

\noindent
{\bf Abstract: }This is a non-breaking proposal to the C++ standard that aims
to reduce allocator complexity, support realtime allocation and improve
performance of node-based containers by making a clear distinction between {\it node}
and {\it array} allocation in the \texttt{std::allocator\_traits} interface.
Two new member functions are proposed, \texttt{allocate\_node} and
\texttt{deallocate\_node}. We also propose that the container node type should
be exposed to the user. A prototype implementation is provided.

\vfill
%\newpage
%\twocolumn
%\linenumbers
\begin{flushright}
\noindent
{\it Size management adds undue difficulties \\
     and inefficiencies to any allocator design} \\
A. ALEXANDRESCU \\
\medskip
{\it }
\end{flushright}
\medskip

\newpage
\tableofcontents

\newpage
\section{Introduction}
\textsc{The importance} of linked data structures in computer science, like
trees and linked lists, cannot be over-emphasised, yet, in the last couple of
years it has become a common trend in C++ to move away from such data
structures due to their sub-optimal memory access patterns \cite{middleditch,
chandler, meyers}.  In fact, many people today prefer to use the flat
alternatives and pay $O(n)$ insertion time, than $O(1)$ at the cost of memory
fragmentation and unpredictable performance loss.  

We believe in fact, that the ``{\it Don't pay for what you don't use}" premise
is not being met on node-based containers due to the restrictive
array-oriented allocator interface. This proposal tries to fix what the author
believes to be the root of problem: {\it The lack of distinction between node
and array allocation}.  We propose here a complete split between these two
allocation techniques by means of a non-breaking addition to
\texttt{std::allocator\_traits}.

\subsection{Node allocation}
%\medskip
%\noindent
%{\bf How simple is node allocation}.
Node allocation is one of the simplest allocation techniques
available, but yet a very powerful one. The simplicity
comes from the fact that allocation and deallocation reduces to pushing and
popping from the linked list of nodes, like shown in the code below.
It is powerful because it is very fast, perform in hard-real-time
and causes minimal memory fragmentation.
\begin{lstlisting}
pointer allocate(std::size_t /*can't handle n*/)
{
  pointer q = avail; // The next free node
  if (avail)
    avail = avail->next;

  return q;
}

void deallocate(pointer p, std::size_t /*can't handle n*/)
{
  p->next = avail;
  avail = p;
}

\end{lstlisting}

The reasons why this allocation technique is not fully supported in C++ is
related to the current {\it array} oriented interface of allocators.  The
member function \texttt{allocate(n)} may be called with $n \ne 1$, meaning that
the allocator now has to implement array allocation strategies instead of much
more simple and efficient node allocation.

Before going into more details, let us see with a use case,
how this proposal provides better solution over the current
array oriented interface.

\subsection{Use case}
%\medskip
%\noindent
%{\bf Use case}.
The example below uses the allocation technique from the previous section to
write code that is fast, simple and uses the minimum amount of memory. A linked
list served with a couple of nodes allocated on the stack
%\medskip
\begin{lstlisting}
  using alloc_t = rt::node_allocator<int>;
  using node_type = typename std::list<int, alloc_t>::node_type;

  // Buffer for 100 elements.
  std::array<char, 100 * sizeof (node_type)> buffer = {{}};
  alloc_t alloc(buffer);

  std::list<int, alloc_t> l1(alloc);
  // Inserts elements. Allocation and deallocation implemented
  // with 6 lines of code.
  l1 = {27, 1, 60};
  ...
\end{lstlisting}

Some of the features of this code are
\begin{itemize}

\item It uses the container node type, to calculate the minimum amount of
memory to support 100 elements in the list. As a consequence the buffer is
compact, improving cache locality and causing minimal fragmentation.

\item The allocator knows it is doing node allocation and does not make the
node size bigger to store bookkeeping information. {\it You do not pay for
space you do not use}. 

\item Very simple and fast allocator where allocation and deallocation
translates into only a couple of pointers assignments. No array allocation
strategy had to be implemented.
%\item {\it It cannot be written portably in C++}.
\end{itemize}

The reasons why we cannot write this code in current C++ will
be better explained below, but shortly said

\begin{itemize}

\item The allocator {\it has to} provide array allocation since 
\texttt{allocate(n)} may be called with $n \ne 1$, as a result
the allocator gets unnecessarily complicated and the size of the
buffer to support 100 elements is not anymore clear since it depends
on the array allocation strategy/algorithm.

\item The container node type and therefore its size is unknown.

\end{itemize}

\subsection{Exposing the node type}

In current C++, there is no straightforward way of knowing the size of the node
the allocator will serve.  At runtime it is known only when the rebound
allocator instance is constructed, which occurs when the container is
constructed. It is a tricky to use this information. As shown in the example
above the user may want to use it to pre-allocate space for a
certain number of elements.

Another situation where the node type is useful is when implementing node
allocators for unordered containers. Usually, unordered containers rebind
twice and there is no way of knowing which rebound type is used for array or
node allocation. Once the node type is exposed the allocator can be specialized
for the desired type, offering node allocation functions accordingly.
%(see appendix \ref{saferalloc}).

The difficult in exposing the node type is the recursiveness of the problem.
The node type is not known until the container type is known, which in turn depends
on the allocator type to be defined and the allocator type cannot be
defined before the node type is known.

At a first glance we may quite naturally demand the node type to be independent
of the container and of the allocator, however, due to the support for fancy pointers
in c++, the following cannot be implemented in general
\begin{lstlisting}
// Cannot always hold for general A1 nd A2.
static_assert( std::is_same< std::list<T, A1>::node_type
                           , std::list<T, A2>::node_type>, "");
\end{lstlisting}

The solution we propose here is to offer a rebind structure in the node type,
so that the specialization can get rid of the allocator pointer type, that
the node type happened to be defined with.  In other words, we can get the
node type from a container defined with any allocator and rebind to a node
with a different pointer type. Recursiveness is bypassed this way.

%We require that the exposed node type support SCARY 
%initialization \cite{scary}. The following should hold
%\medskip
%\begin{lstlisting}
%using node_type1 = std::set<int, C1, A1>::node_type;
%using node_type2 = std::set<int, C2, A1>::node_type;
%using node_type3 = std::set<int, C1, A2>::node_type;
%
%static_assert(std::is_same<node_type1, node_type2>::value, "");
%static_assert(std::is_same<node_type1, node_type3>::value, "");
%static_assert(std::is_same<node_type2, node_type3>::value, "");
%\end{lstlisting}


\subsection{Further considerations}
The influence of fragmentation on performance is well known on the C++
community and subject of many talks in conferences, therefore I am not going to
repeat results here. The interested reader can
refer to \cite{chandler, meyers} for example.

The split between node and array allocation has been successfully implemented
in the Boost.Container library, but the mechanism is based on C++03 instead of
\texttt{std::allocator\_traits}.

For an allocator that explores features proposed here, please see the
project \cite{rtcpp}. For a general talk on allocators and why size management
is a problem \cite{alexandrescu}. For related proposal, please see
\cite{prop1}.  For an alternative approaches to support node allocation, please
see appendix \ref{alternative}.

\section{Motivation and scope}

\textsc{Given the} popularity of the standard allocator, it is important to give reasons
why it should be avoided as a first option for node allocations. I will focus
on the use case given above, where we want to serve an
\texttt{std::list<int>} with a certain number of nodes.

\subsection{Why avoid the standard allocator}

\begin{enumerate}

\item Nodes go necessarily on the heap. For only 100 elements I would preferably
use the stack.

\item The node size is small ($\approx 20$bytes), it is not recommended
allocating them individually on the heap. Fragmentation begins to play a role
if I have many lists or bigger $n$ (see benchmarks below).

\item  Each heap allocation is an overhead: all the code inside malloc, plus
system calls and allocation strategies. (I only need 20 bytes of space for a
node!). Most importantly, the standard allocator does not
know we are doing node allocations and cannot optimize it.

\item Unknown allocated size. Does it allocate more space to store information
needed by the algorithm? How much memory I am really using?

\end{enumerate}

All this is overkill for a simple list with a couple of elements. When the number
of elements gets bigger and the nodes go to the heap, the situation gets much
worser for standard allocator or any custom allocator that implements array allocation
strategies. This is the topic of the next section.

\subsection{Benchmarks}

In the previous section we gave some motivation on why one should avoid
the standard allocator, but what about a custom allocator? To test how much
improvement we can get with custom allocators I tested my own non-optimized
implementation of a node allocator against allocators shipped with GCC.
The graphs can be seem below. The node allocator has never been slower,
in fact, it was most of the time faster than any other fine-tuned allocator.

\begin{figure}[ht]
    \centering
    \includegraphics[scale=1]{fig/node_alloc_bench.pdf}
    \caption[Benchmark]
    {Source code can be found in \cite{rtcpp}.}
    \label{fig::bench}
\end{figure}

%\begin{figure}[ht]
%    \centering
%    \subfloat[]{ \includegraphics[scale=0.5]{fig/std_list_bench.png} }
%    \subfloat[]{ \includegraphics[scale=0.5]{fig/std_set_bench.png} }
%        \\
%    \caption[Benchmarks]
%    {Never slower than blah.}
%    \label{fig::bench}
%\end{figure}


\subsection{General motivations}

Let us see some general motivations on why support for node allocation is
desirable

\begin{enumerate}

\item Support the most natural and one of the fastest allocation
scheme for linked data structures. In \texttt{libstd++} and
\texttt{libc++} for example, it is already possible (by chance) to use
this allocation technique, since $n$ is always $1$ on calls of
\texttt{allocate(n)}.

\item Node-based containers do not manage allocation sizes but
unnecessarily demand this feature from their allocators, with a cost
in performance and overall allocator complexity.

%(Unordered associative containers use
%sized allocations in addition to node allocation, which means they
%need the sized version of \texttt{allocate} as well, but for purposes
%other than node allocation).

\item Support hard-realtime allocation for node-based containers
through pre-allocation and pre-linking of nodes. This is highly
desirable to improve C++ usability in embedded systems.

\item State of the art allocators like \texttt{boost::node\_allocator}
\cite{boost} achieve great performance gains optimizing for the $n = 1$ case. 

\item Avoid wasted space behind allocations. It is pretty common that
allocators allocate more memory than requested to store information
like the size of the allocated block.

\item Keep nodes in as-compact-as-possible buffers, either on the
stack or on the heap, improving cache locality, performance and making
them specially useful for embedded programming.

\end{enumerate}

\section{Impact on the Standard} \label{impact}

\textsc{The following} additions are required in the standard.

\subsection{New \texttt{std::allocator\_traits} members}

We require the addition of two new member functions and
a typedef in \texttt{std::allocator\_traits} as follows

%\newpage
%\medskip
\begin{lstlisting}
template<class Alloc>
struct allocator_traits {
  // Equal to Alloc::node_allocation_only if present,
  // std::false_type otherwise. Array allocation with
  // allocate(n) is ruled out if it is std::true_type.
  using node_allocation_only = std::false_type
  // Calls a.allocate_node() if present otherwise calls
  // Alloc::allocate(1). Memory allocate with this function
  // must be deallocated with deallocate_node.
  pointer allocate_node(Alloc& a);
  // Calls a.deallocate_node(pointer) if present otherwise
  // calls Alloc::deallocate(p, 1). Can only be used with
  // memory allocated with allocate_node.
  void deallocate_node(Alloc& a, pointer p);
};
\end{lstlisting}
%The behaviour of these new members is better explained in section \ref{impact}.
These additions provide the following options inside node-based
containers
\begin{enumerate}
\item {\bf Array allocation only}.
This is the {\it status quo}. Libraries can continue to call
\texttt{allocate(n)} if they want, but since the majority of implementations
use $n = 1$, they may be simply implemented with
\texttt{allocate\_node()}, regardless of whether the allocator provides this
function or not. The implementation of \texttt{allocate\_node()} in the\\
\texttt{std::allocator\_traits} falls back to \texttt{allocate(1)} when the
allocator does not provide one.

\item {\bf Node allocation only}.
In this case, the user is required to set the typedef \texttt{node\_allocation\_only}
to \texttt{std::true\_type} in the allocator and provide \texttt{allocate\_node()}. The user is
not required to provide \texttt{allocate(n)}.
\item {\bf Array and node allocation together}. It is possible to use
both array {and} node allocation when the user provides \texttt{allocate\_node}
and sets \texttt{node\_allocation\_only} to \texttt{std::false\_type}.
I am unaware if this option is useful.
\end{enumerate}

\subsection{The node type}

We require the following node interface on node based containers

%\newpage
\begin{lstlisting}
template <class T, class Ptr>
struct node_type {
  using value_type = T;
  using pointer = // Usually taken from std::pointer_traits<Ptr>
  template<class U, class K>
  struct rebind { using other = node_type<U , K>; };
  // ... implementation details
};
\end{lstlisting}

We also require the node type to be independent of the container
with the exception of the allocator type, for example, the following code should compile.

\newpage
\begin{lstlisting}
  using set_type1 = rt::set<T, C1, A1>;
  using set_type2 = rt::set<T, C2, A2>;

  using pointer = // Arbitray pointer type.
  using node_type1 =
    typename set_type1::node_type::template rebind<T, pointer>;
  using node_type2 = 
    typename set_type2::node_type::template rebind<T, pointer>;
  static_assert(std::is_same<node_type1, node_type2>::value,"");
\end{lstlisting}

All node-based containers are affected: \texttt{std::forward\_list},
\texttt{std::list}, \texttt{std::set}, \texttt{std::multiset},
\texttt{std::unordered\_set}, \texttt{std::unordered\_multiset},
\texttt{std::map}, \texttt{std::multimap},
\texttt{std::unordered\_map}, \texttt{std::unordered\_multimap}

\section{Acknowledgment}

\textsc{I would} like thank people that gave me any kind of feedback: Ville Voutilainen,
Nevin Liber, Daniel Gutson, Alisdair Meredith. Special thanks go
to Ion Gaztañaga for suggesting important changes in the original design and
David Krauss for suggesting other approaches.

\begin{thebibliography}{9}

  \bibitem{middleditch} Sean Middleditch, \url{http://www.open-std.org/jtc1/sc22/wg21/docs/papers/2015/p0038r0.html}
  \bibitem{chandler} Chandler Carruth, {\it Efficiency with Algorithms, Performance
  with Data Structures} (\url{https://www.youtube.com/watch?v=fHNmRkzxHWs})
  \bibitem{meyers} Scott Meyers, {\it Cpu Caches and Why You Care} (\url{https://www.youtube.com/watch?v=WDIkqP4JbkE})
  \bibitem{boost} \url{http://www.boost.org/doc/libs/1_58_0/boost/container/node_allocator.hpp}
  \bibitem{prop1} Ion Gazta\~ naga, \url{http://www.open-std.org/jtc1/sc22/wg21/docs/papers/2006/n2045.html}
  \bibitem{rtcpp} \url{https://github.com/mzimbres/rtcpp}
  \bibitem{alexandrescu} Andrei Alexandrescu, {\it std::allocator Is to Allocation what
  std::vector Is to Vexation} (\url{https://www.youtube.com/watch?v=LIb3L4vKZ7U})
  \bibitem{embedded} \url{http://www.open-std.org/pipermail/embedded/2014-December/000335.html}
  \bibitem{proplist} \url{https://groups.google.com/a/isocpp.org/forum/#!topic/std-proposals/ccwOpTxM_xE}
  \bibitem{scary} \url{http://www.open-std.org/jtc1/sc22/wg21/docs/papers/2009/n2980.pdf}

\end{thebibliography}

\appendix

\section{Alternative approaches} \label{alternative}

\textsc{There are} other possible approaches to support node allocation that are worth knowing
of.  I will describe them here, so that the committee can compare them.

\subsection{\texttt{allocate(n)} with $n = 1$}

%\medskip
%\noindent
%{\bf Ensure \texttt{allocate(n)} is called with $n = 1$}.
This seems the easiest
way to perform node allocation. Once the standard guarantees $n$ will be always $1$,
there is no more need to provide new node allocation functions for node-based containers. The
parameter $n$ can be simply ignored. The \texttt{allocate} and \texttt{deallocate}
can be implemented in terms of node-allocation-only functions, for example
\medskip
\begin{lstlisting}
pointer allocate(std::size_t /* n is ignored */)
{
  return allocate_node();
}

void deallocate(pointer p, std::size_t /* n is ignored */)
{
  deallocate_node();
}
\end{lstlisting}

The problem with this approach is that it prevents array allocation inside
node-based containers, which means it can be viewed as a narrowing of the
current interface.

\subsection{Provide a \texttt{constexpr max\_size()} that returns 1}

%\medskip
%\noindent
%{\bf Provide a \texttt{constexpr max\_size()} that returns 1}.

This scheme
can achieve the same goals as my main proposal and does not require any
addition to \texttt{std::allocator\_traits}. Libraries should check if
\texttt{max\_size()} can be evaluated at compile time and take appropriate
action i.e. ensure \texttt{allocate(n)} is always called with $n = 1$.
I did not adopted it due to some disadvantages I see with it

\begin{enumerate}
\item It does not make containers implementation simpler.

\item Function names should reflect that array and node allocation have
different semantics, apart from the storage size. If memory expansion (realloc)
is added in the future, it should only work with storage allocated with
\texttt{allocate(n)} but not with storage allocated for nodes. This allows node
allocations to avoid extra bookkeeping data to mark the storage as
non-expandable.

\item It requires the user to specialize \texttt{std::allocator\_traits} to
provide a \texttt{constexpr max\_size()} since the default is not
\texttt{constexpr}. This is not bad but I prefer to avoid it if I can.

\item Other static information like \texttt{propagate\_on\_container\_copy\_assignment}, etc,
are provided as typedef so I prefer to keep the harmony.

\item It sounds more like a hack of the current allocator interface to achieve
node allocation than a full supported feature.

\end{enumerate}

\newpage

%\section{Node type and safer allocators} \label{saferalloc}
%
%\medskip
%\begin{lstlisting}
%  // Simulates the container node type. For example
%  // std::unordered_set<int>::node_type. I am using
%  // an arbitrary pointer type as an example.
%  using node_type1 = node<int, char*>;
%
%  // User node allocator configured for node_type1.
%  using alloc_type = rt::node_allocator<int, node_type1>;
%
%  // Container specific node type. Known to the user only after
%  // the container type itself is known. For example
%  // std::unordered_set<int, ... , alloc_type>::node_type.
%  // It is equal to node_type1 except (possibly) by the pointer
%  // type.
%  using node_type2 = node<int, double*>;
%
%  // Buffer for 100 elements
%  std::array<char, 100 * sizeof (node_type2)> buffer = {{}};
%
%  alloc_type alloc(buffer); // User allocator instance.
%
%  // Not node_type2, array allocation is fine.
%  auto a = alloc.allocate(10); // Ok
%
%  // Error, node allocation not enabled for int.
%  auto b = alloc.allocate_node(); // Compile time error
%
%  // The following lines occurs inside the unordered container.
%
%  // Rebinds to get an allocator for node_type2.
%  using node_alloc_type =
%    typename alloc_type::template rebind<node_type2>::other;
%
%  node_alloc_type node_alloc(alloc);
%
%  // Error, array allocation not available for node_type.
%  auto c = node_alloc.allocate(10); // Compile time error
%
%  // Ok, node allocation enabled.
%  auto d = node_alloc.allocate_node(); // Ok
%
%  // Rebinds to the unordered container internal array
%  // allocation type. 
%  using array_alloc_type =
%    typename alloc_type::template rebind<float*>::other;
%
%  array_alloc_type array_alloc(alloc);
%
%  // Ok, Not the node type, array allocation enabled.
%  auto e = alloc.allocate(10); // Ok
%
%  // Error, node allocation is not enabled for float*.
%  auto f = alloc.allocate_node(); // Compile time error.
%\end{lstlisting}

\end{document}

